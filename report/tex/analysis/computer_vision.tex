\section{Computer Vision}
Computer vision is the field of study relating to how a computer can perceive the world as human vision does through videos and still images. There are several challenges when working with computer vision, such a discerning front and background in a picture, identifying an object and verifying the integrity of it across frames in a video, capturing the shape of a 3D object from pictures etc. and it is a field in progress.

Today computer vision is being used for, among other things, 3D construction of buildings from aerial images, match movie i.e. the act of merging computer graphics to live footage so it appears realistic (e.g. in the movie Jurassic Park), fingerprint recognition and surveillance of pools for drowning people \cite{szeliski11}.

The use of computer vision is imperative for the Kinect or webcam to work, seeing as it relies on its two cameras. Thus if a camera is chosen, computer vision will undoubtedly be a big part of the sensor network, helping with target identification and tracking.

There are several libraries available to facilitate computer vision, such as Matrox Imaging Library \cite{matrox} and NI Vision \cite{nivision}, but both of these are commercial products and only offer free 30-day-trials, which will not last the duration of the project period.
 
Another option is libCVD \citep{libcvd}, Cambridge video dynamics, which is free and open-source, however it appears that the most recommended library is \ac{opencv} \cite{opencv}, which will be described in further detail in the following section.