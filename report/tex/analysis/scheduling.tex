\section{Scheduling}
In a program with concurrency, the exact order tasks are executed in is not completely specified. Different methods can be used to ensure that the correct task is running at the correct time. It is possible to use semaphores to restrict one task to use any shared data at one time, or a message can be used to make a task wait for a computation to finish. These methods, however, do not stop tasks from interleaving, which makes it possible for tasks to block other tasks from accessing a shared resource when it is needed. Some tasks can be more important than others. If a non-important task blocks a safety-critical task, because they use the same shared resource, it can make the critical task fail, and in a worst case scenario cause harm to someone. To stop this from happening,  scheduling is needed\cite{rts-book}.

There are three types of tasks, periodic, sporadic and aperiodic tasks. Periodic tasks have a set period between their runtimes, aperiodic tasks do not. Aperiodic tasks can run multiple times in succession, without breaks in between, this makes them almost impossible to schedule. Sporadic tasks are somewhere in between periodic and aperiodic tasks, they do not have a known period, but they have a minimum period between each run.

Scheduling ensures that a concurrent program produces the expected outcome in all cases and that it complies with all of its deadlines.