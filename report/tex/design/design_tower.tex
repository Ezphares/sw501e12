\section{Tower design}
\label{turretdesign}
As mentioned in our section on requirements, our projectile launcher has some firm requirements. It needs to be able to detect, aim and fire at a moving target. 
From a design perspective we need two distinct properties namely, horizontal and vertical rotation. There are of course many different ways of achieving both, but in our case the simplest thing would be to have one servo motor handle vertical rotation and another servo motor to handle horizontal rotation. 

We started with the model depicted in figure~\ref{first_model}, with an alternative firing mechanism which is a special LEGO competition cannon as seen in figure~\ref{competition_cannon}. 

\begin{figure}[hptb]
  \centering
    \includegraphics[width=0.5\textwidth]{img/Step114.png}
  \caption{Initial model for the projectile launcher}
  \label{first_model}
\end{figure}

But the construction was too top heavy thus making it difficult to achieve a precise and reliable rotation. We tried, but were unable to find an acceptable solution. The main problem was that a single servo was responsible for rotating the everything except the base, and the servo even had to handle its own weight. 

\begin{figure}[hptb]
  \centering
    \includegraphics[width=0.5\textwidth]{img/competition_cannon.png}
  \caption{Our projectile firing mechanism}
  \label{competition_cannon}
\end{figure}

Figure~\ref{final_model} shows the final design of the projectile launcher. We focused on a relatively flat design, minimizing the weight on the rotating turret. We also refactored vertical raising mechanism to a minimal set of components and the horizontal rotation was refactored in such a way that we could remove the servo from the turret. 

The rotation was restricted to a fixed set of degrees. This decision was based on the hardware analysis where we discovered that the Kinect's field of view covers 57 horizontal degrees and 34 vertical degrees. 

The turret itself had to refactored to only contain the parts necessary for the construction to meet it's requirements, this meant that the NXT and the Kinect did not have to be placed on the turret, but rather behind and in front of the turret. We could place the servo that handled horizontal rotation on a structure that stood next to the turret. And lastly we needed some way of configuring the starting position of the turret, in order to be absolutely sure that the turret was facing in the same direction on every initialization. So we placed a pressure sensor next to the turret and mounted a beam onto the turret that would hit the sensor at a certain angle. It should also be noted that the competition cannon in figure~\ref{competition_cannon} is not included in LEGO Digital Designer, which we used to make the model, and therefore the cannon is missing from the model in figure~\ref{final_model}.
\begin{figure}[hptb]
  \centering
    \includegraphics[width=0.5\textwidth]{img/design_turret4.png}
  \caption{Final model of the projectile launcher.}
  \label{final_model}
\end{figure}

\subsection{Placing the Kinect} % (fold)
\label{sub:placing_the_kinect}
There are several options with regards to the placement of the Kinect in relation to the Projectile Launcher. The ideal solution would probably be to place the Kinect right onto a rotating the turret with the possibility of 360 degree rotation, this would enable the Projectile Launcher to hit targets all around it. However this proved inefficient with regards to weight distribution and stability.

Another approach would be to place the Kinect behind or next to the turret, with the turret within the Kinect's field of vision. This would enable the Kinect to calculate the target's position and speed in relation to the turret. The problem is that the turret would also obstruct the Kinect's field of vision, resulting in irregularities and might even lead to false calculations.

A third approach, and the one we implemented, is to place the Kinect right in front of the turret, directly under the actual cannon. In this case we only have to calculate the height difference between the Kinect and the cannon. The Kinect can only see a limited amount of degrees on the horizontal axis, and we have configured the turret to be able to rotate accordingly. Thus fulfilling the requirements.
%subsection placing_the_kinect (end)