\section{Scheduling}

As seen in \autoref{nxtdesign} the embedded application should be split up in several tasks.
The most important parts of determining scheduling in this case, is when the task should run
(period), when it should be done (deadline), and how long it actually takes (cost).

Since the LEGO Technic competition cannon requires a manual reset, it was decided that
the program should be scheduled to run once, and that a manual reset of the software is
acceptable as well, since access to the NXT is as easy as access to the cannon.

The tasks are summarized below:

\begin{itemize}
  \item{Reset}\\
  Resets the turret to a known position. This should be released once, as the application starts.
  It has no fixed deadline, as it is part of initialization, and the time it takes is
  simply a necessity for the system to work. The time it takes is determined by the direction
  of the turret before the system starts.
  \item{Kinect}\\
  Reads from the USB, and processes data if any is received. The nxtOSEK specifications dictate
  that the USB is processed every millisecond, so that will be the period of this task. That also
  means it has a deadline of 1 ms, lest the USB connection is timed out. The time taken to complete
  this is slightly longer when data is received, as some processing takes place immediately.
  \item{Aim}\\
  Predicts target movement and aligns the turret to the target point. This should be run once,
  when data is received in the Kinect task. It's deadline is "before the turret has to fire" - this
  is described in greater detail, below. The time it takes is determined by how far the turret has
  to turn to hit it's target.
  \item{Fire}\\
  Fires the projectile. Should be run once, at a set time after receiving data, depending on the
  prediction time, and the distance between the cannon and the target. The deadline is before 
  the target passes, and it takes a fixed time.
\end{itemize}

\subsection{Turret firing time}
The point in time at which the turret has to fire can be described as\\
\begin{math}
Moment_{fire} = Moment_{received} + Time_{prediction} - ( Distance_{target} / Speed_{projectile} )
 - Time_{delay}
\end{math}
Where:
\begin{itemize}
  \item $Moment_{fire}$ is the absolute time at which the Fire task is released.
  \item $Moment_{received}$ is the absolute time at which data was received on the USB.
  \item $Time_{prediction}$ is how far into the future motion is predicted.
  \item $Distance_{target}$ is how far from the cannon the target is PREDICTED to be.
  \item $Speed_{projectile}$ is the average speed of the projectile moving to the target.
  \item $Time_{delay}$ is the fixed delay in the system, comprised of the time to send data from the
  Kinect to the NXT, and the time after starting the fire task until the projectile is actually released.
\end{itemize}
The only one of these parameters that can be changed as desired is $Time_{prediction}$. However
there is another dynamic parameter, $Distance_{target}$ that needs to be taken into account.
There is two options for this:
\begin{enumerate}
  \item Increasing $Time_{prediction}$ based on the distance. This means that $Moment_{fire}$
  always have a offset from $Moment_{Received}$, and as that keeps the deadline for the Aim
  task, fixed as well.
  \item Keeping $Time_{prediction}$ constant, reducing the time between $Moment_{Received}$
  and $Moment_{fire}$ based on distance. This changes the deadline for Aim, but keeps
  the prediction calculation easier.
\end{enumerate}
The first option would seem best at first, however, initial experiments with the Kinect and
image analysis revealed some imprecisions in determining the target's motion. This means
that a change to prediction time, also makes predictions less accurate, since predicting
farther into the future inflates any such imprecisions. Instead the second option was chosen,
even though this puts an effective maximum distance on the turret, namely the distance where
the time to fire will occur before aiming is done.

\subsection{Method of Scheduling}
nxtOSEK gives us access to a powerful scheduler, which may be overkill for this system,
considering that three of the tasks, Reset, Aim and Fire, should never be released at
the same time. This coincidentally also makes them sporadic, in a real-time context,
which normally whould make them harder to account for than periodic tasks. These
are released as a one-time autorun, instantly by event, and by a dynamic alarm
respectively.

The only periodic task, Kinect, is however scheduled to run each millisecond, and
should be prioritized higher than the other tasks, as well as allow full pre-emption,
to prevent it from timing out.

The simple nature of task timings make for a very predictable system, as long as it's
restrictions are complied with. It also means, however, that these restrictions are
hard, and not complying will make the system do nothing at all.