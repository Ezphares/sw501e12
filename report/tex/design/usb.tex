\section{USB-protocol}
The USB-protocol needs to connect the results from the Kinect to the NXT. The specifications of the USB-protocol is not easily changable, as both the Kinect program and the NXT program is dependent on it, and a change would cause partial rewrites of both programs. It is important for the USB-protocol specifications to be decided early, and not changed.
Furthermore, the NXT has very limited memory, and the USB package size is limited to 64 bytes, the data will have to be as minimal as possible while still retaining all the important information. 

The USB-protocol needs to be able to connect the computer to the NXT, and in the case of no NXT found, it should give an error.
When the NXT is connected, and the Kinect has found an object, the USB-protocol will facilitate the sending of the object-specific data.
More specifically, the data that needs to be sent is the x, y, and z position of the object, and the observed x, y, and z speed of the object.

The data will be sent as an array of signed integers, with each element taking up one space, making the array six items long. This way, it is easy to know which integer contains which information, without using any unnecessary space.