\section{Image Analysis System}

Implementing the design from \autoref{image_analysis_design} did reveal some interesting problems. One of these
was the conversion from a location detected by the Kinect, to a point in a 3D coordinate system. The code involved
in this can be seen in \autoref{fig:kinectconvert}. The functions on the KinectData object are preliminary conversions,
as the raw data is not very useful for position calculations: The $h$ and $v$ values are pixel coordinates, and the $depth$
is received using a formula giving more precise results at close range.

\begin{figure}[hbtp]
\lstinputlisting[language=Python]{listings/convert.py}
\caption{Kinect to Coordinate conversions}
\label{fig:kinectconvert} 
\end{figure}

The $h$ and $v$ values are converted to degrees by first dividing them with the resolution of the received image, getting
a number between 0 and 1, which is then multiplied by the field of view of the Kinect. The depth is converted to
millimeters. At this point the law of sines can be used to find the $x$ and $y$ coordinates of the object, also in
millimeters, and with those, Pythagoras' theorem can be used to find the $z$ coordinate.



