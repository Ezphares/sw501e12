\section{NXT Application}
The application on the NXT is written in C and corresponds closely with the tasks and math described in previous chapters.

\subsection{TaskAim}
Once the final target coordinates have been found, i.e. the predicted position of the target when aiming is done, the motors need to be set to the correct angle to hit. This is done in two separate stages. First the horizontal angle is calculated, then the vertical angle is calculated.

\autoref{lst:aim} is a code snippet documenting the implementation of the mathematical formulae used to aim.
\lstinputlisting[language=C, caption=Horizontal and Vertical Aim, label=lst:aim]{listings/aim.c}

Lines 2-3 determine the prediction time, and creates the position vector for the target after the designated
time has passed. With the position of the target known, it is possible to adjust for the difference in position 
between the Kinect and the cannon, and calculate and set the alarm that releases the fire task (line 6-12).

Aiming horizontally is simply done using the law of sines (line 15-16), but calculating the vertical angle is more difficult, because the trajectory of the projectile has to be taken into consideration. This trajectory depends on the speed of the projectile, the angle fired and the height fired from.

\begin{figure}[htbp]
$$tan_a = {v \text{ \textpm} \text{ } v^2 - 2g(y + {1 \over 2}g(z^2/v^2))^{1 \over 2} \over gz/v}$$
\caption{Calculation of vertical angle to hit target}
\label{fig:vertical-angle}
\end{figure}

\autoref{fig:vertical-angle} shows the formula to find the angle needed to hit a target: An equation is derived which  
includes the velocity of the projectile, $v$, the gravitational acceleration of the Earth $g$, the distance to the
target $z$, and the height of the projectile over the ground $y$. It should be noted that the $z$ in this formula
is actually the depth to the object, and not its $z$-coordinate. This calculation results in two possibles angles
of fire, stored in $degrees\_pos$ and $degrees\_neg$ (line 21-28).

The values are negative if it is necessary to raise the turret above its initial position, therefore max (on line 30)
chooses the one closest to 0\textdegree . This gives the fastest possible aim, as lifting the turret takes a nonzero
amount of time.
