\section{NXT Application}
The application has the following tasks, events etc:

\lstinputlisting[language=C, caption=Declared Objects, label=lst:declaredobjects]{listings/declares.c}

The counter is implemented because this is the only way to keep track of time in the NXTOSEK OS \cite{osek_spec}.
Resources are shared resources that need to be protected using mutual exclusion. The Target struct is implemented as a shared resource because this is the easiest way to share information between tasks.

It should be noted that the last 2 constants are determined by empirical evidence in the analysis. The horizontal relation is the relation between servo motor and the turntable responsible for horizontal movement.

Below is a code snippet documenting the implementation of the mathematical formulas used to aim. The prediction double is set according to the scheduling in \autoref{scheduling}. This is the most advanced math on the NXT brick.

\lstinputlisting[language=C, caption=Horizontal and Vertical Aim, label=lst:aim]{listings/aim.c}

As seen, the implementation is indeed very simple and straightforward given the mathematical formulas explained in the analysis, and this can be said for all tasks in the application. 